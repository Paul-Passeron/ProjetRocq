\documentclass{report}
\usepackage{coqdoc}
\usepackage[utf8]{inputenc}
\usepackage[T1]{fontenc}
\usepackage{hyperref}

\begin{document}

\begin{titlepage}
	\begin{center}
		\vspace*{1cm}
		
		{\Huge \textbf{Preuve formelle mécanisée}}
		
		\vspace{0.3cm}
		{\huge Utilisation du système Rocq}

		\vspace{0.5cm}
		{\LARGE projet 2025-2026}
		
		\vspace{1.5cm}
		
		{\Large \textbf{Paul PASSERON, Jean ROUSSEAU}}
		
		%\vspace{10cm}
		
		\vfill

		\today
		
	\end{center}
\end{titlepage}
 
	    
\tableofcontents 

\chapter{Introduction} 
	
	\section{Préambule}
	
	\begin{itemize}
		\item Le projet a été réalisé sur VSCode (VScoq) d'une part, et sur coqide de l'autre.
		\item Les résultats ont été mis en commun grâce à Github.
		\item Le rapport a été écrit en \LaTeX et le chapitre 2 a été généré grâce à coqdoc.
	\end{itemize}

	\section{Livrables}

	Les livrables de ce projet sont :
	\begin{itemize}
		\item Le sujet : \texttt{projet2025.pdf}
		\item Le rapport : \texttt{rapport.pdf}
		\item Le fichier source : \texttt{projet\_rocq\_passeron\_rousseau.v}
		\item Un fichier makefile pour générer le rapport : \texttt{makefile}
		\item Un fichier \LaTeX contenant le titre du rapport : \texttt{titre.tex}
		\item Un fichier \LaTeX contenant le squelette du rapport : \texttt{rapport.tex}
		\item Un fichier \LaTeX contenant le corps du rapport : \texttt{projet\_rocq\_passeron\_rousseau.tex}
	\end{itemize}

	Le tout est contenu dans une archive [nom de l'archive].tar.gz.

	\section{Mode d'emploi}

	Si pour une raison quelconque le rapport a mal été généré, il est possible d'en faire un nouveau en suivant les étapes suivantes : 
	\begin{enumerate}
		\item Se placer dans le dossier extrait de l'archive.
		\item Faire la commande : \texttt{make clean}
		\item Faire la commande : \texttt{make}
	\end{enumerate}

	Une fois cela fait, un nouveau fichier \texttt{rapport.pdf} devrait être généré.

	\section{Remarque}
	Certaines preuves sont faites sans \textit{bullet points}, notamment les preuves avec deux \textit{goals} dont le premier est trivialement vrai (pas plus de deux ou trois lignes).

\input{coq_body.tex}

\chapter{Conclusion}

	À l'exception des questions \textbf{4.c} et \textbf{4.d} de \textbf{l'exercice 1}, tout a été prouvé.

\end{document}

